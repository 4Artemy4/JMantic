\section{Ostis}
\label{sec:ostis}
В данном разделе описывается протокол и формат обмена сообщений. Раздел предназначен, в первую очередь, для разработчиков JMantic, и описывает внутренние механизмы работы библиотеки. 

\subsection{Обзор}
Запросы и ответы к sc-machine работают по протоколу WebSocket, и представляют из себя текст в формате json с одним корневым блоком. Каждый запрос содержит идентификатор, тип запроса и полезную нагрузку. Ответ содержит идентификатор, который соответствует идентификатору запроса, статус и полезную нагрузку. 

\subsection{Структура запрос}
\begin{itemize}
\item id --- идентификатор запроса. 
\item type --- тип запроса. Задаёт команду, которую нужно выполнить sc-machine. 
\item payload --- последовательность данных запроса/ответа. Для разных типов запросов/ответов содержание данной секции будут отличаться. Может содержать как один набор данных, так и массив наборов данных. 
\end{itemize}

\subsection{Структура ответ}
\begin{itemize}
\item id --- идентификатор ответа.
\item status --- состояние ответа. Может принимать значения true и false. При успешном выполнении запроса данное поле будет содержать true. В ином случае -  false. Если данный ответ сгенерирован событием sc-machine, данного поля может не быть.
\item event --- флаг ответа. Показывает, сгенерирован ли данный ответ событием. Если да --- поле будет иметь значение true. В ином случае этого поля может не быть или значение равно false.
\item payload --- последовательность данных запроса/ответа. Для разных типов запросов/ответов содержание данной секции будут отличаться. Может содержать как один набор данных, так и массив наборов данных. 
\end{itemize}

\subsection{id}
Идентификатор представляет из себя целое число, которое служит для сопоставления ответа запросу при асинхронном режиме работы. То есть, на запросу с определённым id придёт ответ с таким же id.

Запрос:
\begin{lstlisting}[language=json,firstnumber=1]
{
  "id": 1,
  "type": "request type",
  "payload": {
    ...
  }
}
\end{lstlisting}
Ответ:
\begin{lstlisting}[language=json,firstnumber=1]
{
{
  "id": 1,
  "status": true,
  "event": false, 
  "payload": {
    ...
  }
}
}
\end{lstlisting}


\subsection{type}
Тип запроса --- это строка из набора допустимых запросов. Возможно использовать следующие типы запросов: 
\subsubsection{create\_elements}
create\_elements --- запрос на создание sc-элементов.  Запрос может содержать набор элементов, которые нужно создать, но каждый элемент должен представлять один из трёх допустимых вариантов: 
\begin{itemize}
\item node --- sc-узел;
\item edge --- sc-дуга;
\item link --- sc-ссылка.
\end{itemize}
Каждый элемент должен иметь тип. Допустимые типы описаны в таблице(\ref{all_types}).
В секции payload можно указать как один элемент, так и несколько. В ответе на запрос в секции payload будут указаны адреса созданных объектов. При этом каждый адрес будет соответствовать элементу запроса в том же порядке (первый элемент - первый адрес). 
Пример:
\begin{lstlisting}[language=json,firstnumber=1]
{
  "id": 2,
  "type": "create_elements",
  "payload": [
    {
      "el": "node",
      "type": 1
    },
    {
      "el": "link",
      "type": 2,
      "content": 45.4
    },
    {
      "el": "edge",
      "src": {
        "type": "ref",
        "value": 0
      },
      "trg": {
        "type": "ref",
        "value": 1
      },
      "type": 32
    }
  ]
}
\end{lstlisting}

При создания sc-ссылки можно также указать тип содержимого (см. описание запроса content). 

Пример секции с указанием типа:
\begin{lstlisting}[language=json,firstnumber=1]
{
  "el": "link",
  "type": 2,         
  "content": "data", 
  "content_type": "string"  
}
\end{lstlisting}

\subsubsection{check\_elements}
check\_elements --- запрос на проверку существования sc-элементов. В payload указываются адреса sc-элементов, существование которых необходимо проверить . В секции payload ответа будут указаны числа, которые соответствуют типу sc-элемента в случае существования элемента. Если элемент не найден, в ответе будет указано число 0, иначе будет указан числовой код найденного элемента. Порядок элементов в ответе соответствует порядку элементов в запросе.

Пример  запроса:
\begin{lstlisting}[language=json,firstnumber=1]
{
  "id": 3,
  "type": "check_elements",
  "payload": [
    23123,
    432,
    ...
  ]
}
\end{lstlisting}

Пример ответа: 
\begin{lstlisting}[language=json,firstnumber=1]
{
  "id": 3,
  "payload": [
    32, 
    0,  
    ...
  ]
}

\end{lstlisting}

\subsubsection{delete\_elements}
delete\_elements --- запрос на удаление sc-элементов. Работает аналогично с запросом проверки существования.
Пример:
\begin{lstlisting}[language=json,firstnumber=1]
{
  "id": 4,
  "type": "delete_elements",
  "payload": [
    23123,
    432,
    ...
  ]
}
\end{lstlisting}

\subsubsection{search\_template}
search\_template --- запрос на поиск sc-конструкции по шаблону. Шаблон представляет из себя список, каждый из элементов которого является тройкой элементов, часть из которых известны. В ответе, если шаблон найден, будут подставлены адреса неизвестных ранее элементов. Также можно указать некоторые тройки шаблона как необязательные. Тогда в ответе, если такие тройки не найдены, на их месте будут нули, но остальным тройкам шаблона будут подставлены адреса соответствующих элементов. Это можно сделать указав после блоков элементов тройки дополнительный блок is\_requied, присвоив ему значение false. 

Пример:
\begin{lstlisting}[language=json,firstnumber=1]
{
  "id": 5,
  "type": "search_template",
  "payload": [
    [
      {
        "type": "addr",
        "value": 23123  
      },
      {
        "type": "type",
        "value": 32,    
        "alias": "_edge1"
      },
      {
        "type": "type",
        "value": 2, 
        "alias": "_trg"  
      }
    ],
    [
      {
        "type": "addr",
        "value": 231342
      },
      {
        "type": "type",
        "value": 2000,
        "alias": "_edge2"
      },
      {
        "type": "alias",
        "value": "_edge1" 
      },
      {
        "is_required": false
      }
    ],
    ...
  ]
}
\end{lstlisting}
Ответ для данного запроса будет выглядеть следующим образом:
\begin{lstlisting}[language=json,firstnumber=1]
{
  "id": 5,
  "payload": {
    "aliases": {
      "trg": 2,
      "edge1": 1,
      "edge2": 4
    },
    "addrs": [
      [23123, 412, 423, 231342, 282, 412], 
      [23123, 6734, 85643, 231342, 4234, 6734],
      [23123, 7256, 252, 0, 0, 0],
      ...
    ]
  }
}
\end{lstlisting}

Также существует возможность указать шаблон поиска с помощью текста на языке SCs.
Пример:
\begin{lstlisting}[language=json,firstnumber=1]
{
  "id": 6,
  "type": "search_template",
  "payload": "person _-> .._p (* _=> nrel_email:: _[] *);;"
}
\end{lstlisting}

\subsubsection{generate\_template}
generate\_template --- запрос на создание sc-конструкций по шаблону. Работает по схожему с поиском принципу. 

Пример запроса:
\begin{lstlisting}[language=json,firstnumber=1]
{
  "id": 7,
  "type": "generate_template",
  "payload": {
    "templ":
    [
      {
        "params": [
          {
            "type": "addr",
            "value": 23123  
          },
          {
            "type": "type",
            "value": 32,    
            "alias": "_edge1"
          },
          {
            "type": "type",
            "value": 2,     
            "alias": "_trg" 
          }
        ]
      },
      {
        "params": [
          {
            "type": "addr",
            "value": 231342
          },
          {
            "type": "type",
            "value": 2000,
            "alias": "_edge2"
          },
          {
            "type": "alias",
            "value": "_edge1"  
          }
        ]
      },
      ...
    ],
    "params": {
      "_trg": 564
    }
  }
}
\end{lstlisting}
Ответ на запрос:
\begin{lstlisting}[language=json,firstnumber=1]
{
  "id": 7,
  "payload": {
    "aliases": {
      "_trg": 2,
      "_edge1": 1,
      "_edge2": 4
    },
    "addrs": [23123, 4953, 564, 231342, 533, 4953]
  }
}
\end{lstlisting}

Как и в случае поиска, шаблон может быть представлен текстом на языке SCs:
\begin{lstlisting}[language=json,firstnumber=1]
{
  "id": 8,
  "type": "generate_template",
  "payload": {
    "templ": "person _-> .._p (* _=> nrel_email:: _[test@email.com] *);;",
    "params": {
      ".._p": 5314
    }
  }
}
\end{lstlisting}

\subsubsection{events}
events --- запрос на подпись на события или удалить существующие подписки. Тип события представляет из себя строку и указывается в разделе payload. Существует 6 видов событий, на которые можно подписаться:
\begin{itemize}
    \item add\_outgoing\_edge --- генерирует событие, при создании выходящей sc-дуги из наблюдаемого sc-элемента
    \item add\_ingoing\_edge --- генерирует событие, при создании входящей sc-дуги в наблюдаемый sc-элемент;
    \item remove\_outgoing\_edge --- генерирует событие, при удалении выходящей sc-дуги из наблюдаемого sc-элемента
    \item remove\_ingoing\_edge --- генерирует событие, при удалении входящей sc-дуги в наблюдаемый sc-элемент;
    \item content\_change --- генерирует событие при изменении содержания у наблюдаемой sc-ссылки;
    \item delete\_element --- генеирует событие при удалении наблюдаемого sc-элемента. 
\end{itemize}
Чтобы подписаться на событие, необходимо в разделе payload указать раздел create, в котором перечислить все события, на которые необходимо подписаться. Сам блок описания вида события представлен типом события из списка а также адресом sc-элемента, который будет генерировать ответы при событиях. 

Для удаления подписки на событие необходимо в блоке payload создать блок delete, в котором перечислить идентификаторы запросов, которыми была осуществлена подписка. 

Пример:
\begin{lstlisting}[language=json,firstnumber=1]
{
  "id": 9,
  "type": "events",
  "payload": {
    "create": [
      {
        "type": "add_output_edge",  
        "addr": 324             
      }
    ],
    "delete": [
      2, 4, 5
    ]
  }
}
\end{lstlisting}

\subsubsection{keynodes}
keynodes --- запрос на получение и редактирование ключевых узлов (keynodes).
С помощью данной команды можно найти адрес узла в памяти sc-machine по его идентификатору или создать новый идентификатор(идентификаторы в рамках sc-machine уникальны). 
\begin{itemize}
\item find --- поиск sc-элемента по идентификатору 
\item resolve --- поиск sc-элемента по идентификатору. Если идентификатора не существует, то будет создан новый, адрес которого будет содержаться в ответе. При этом, кроме идентификатора, в запросе нужно указать тип элемента (должен быть узлом).
\end{itemize}

Пример запроса:
\begin{lstlisting}[language=json,firstnumber=1]
{
  "id": 11,
  "type": "keynodes",
  "payload": [
    {
      "command": "find",
      "idtf": "any system identifier that NOT exist"
    },

    {
      "command": "find",
      "idtf": "any system identifier that exist"
    },

    {
      "command": "resolve",
      "idtf": "NOT exist",
      "elType": 1
    },

    {
      "command": "resolve",
      "idtf": "exist",
      "elType": 1
    }
  ]
}
\end{lstlisting}

Пример ответа:
\begin{lstlisting}[language=json,firstnumber=1]
{
  "id": 11,
  "status": true,
  "payload": [
    0,
    321,
    435,
    324
  ]
}
\end{lstlisting}


\subsubsection{content}
content --- запрос на получение и редактирование содержимого sc-ссылок. Sc-ссылки могут хранить один из 4 типов: 
\begin{itemize}
    \item int --- целое число;
    \item float --- вещественное число;
    \item string --- строка;
    \item binary --- бинарные данные.
\end{itemize}

Для работы с содержимым sc-ссылке используются 2 блока: get и set. Блок set служит для изменения содержимого существующей sc-ссылки, и принимает тип хранимого значения, само значение, адрес sc-ссылки в sc-памяти. Блок get служит для получения содержимого sc-ссылки, принимая адрес этой ссылки в sc-памяти. В ответе на get-блок будет содержаться значение хранимого типа(либо null, если содержимого нет) и сам тип. Ответ на set-запрос представляет из себя одно слово --- true, если значение записалось и false если запись не удалась.  

Пример запроса:
\begin{lstlisting}[language=json,firstnumber=1]
{
  "id": 10,
  "type": "content",
  "payload": [
    {
      "command": "set",
      "type": "int",  
      "data": 67,     
      "addr": 3123   
    },
    {
      "command": "get",
      "addr": 232       
    },
  ]
}
\end{lstlisting}

Ответ на предыдущий запрос будет следующим: 
\begin{lstlisting}[language=json,firstnumber=1]
{
  "id": 10,
  "payload": [
    true,      
    {
      "value": 56.7,  
      "type": "float"
    },
  ]
}
\end{lstlisting}


\subsection{Таблица базовых типов sc-элементов}

\begin{figure}[H]
    \centering
    \begin{tabular}{  | l | c | c | }
		\hline
		Тип & Шестнадцатеричный & Десятичный  \\
        \hline
		sc\_type\_node  & 0x1 & 1  \\
        \hline
		sc\_type\_link  & 0x2 & 2  \\
        \hline
		sc\_type\_uedge\_common  & 0x4 & 4  \\
        \hline
		sc\_type\_dedge\_common  & 0x8 & 8  \\
        \hline
		sc\_type\_edge\_access  & 0x10 & 16  \\
        \hline
		sc\_type\_const  & 0x20 & 32  \\
        \hline
		sc\_type\_var & 0x40 & 64  \\
        \hline
		sc\_type\_edge\_pos & 0x80 & 128  \\
        \hline
        sc\_type\_edge\_neg & 0x100 & 256  \\
        \hline
        sc\_type\_edge\_fuz & 0x200 & 512  \\
        \hline
        sc\_type\_edge\_temp & 0x400 & 1024  \\
        \hline
        sc\_type\_edge\_perm & 0x800 & 2048  \\
        \hline
        sc\_type\_node\_tuple & 0x80 & 128  \\
        \hline
        sc\_type\_node\_struct & 0x100 & 256  \\
        \hline
        sc\_type\_node\_role  & 0x200 & 512  \\
        \hline
        sc\_type\_node\_norole & 0x400 & 1024  \\
        \hline
        sc\_type\_node\_class & 0x800 & 2048  \\
        \hline
        sc\_type\_node\_abstract & 0x1000 & 4096  \\
        \hline
        sc\_type\_node\_material  & 0x2000 & 8192  \\
        \hline
        sc\_type\_arc\_pos\_const\_perm & 0x8B0 & 2224  \\
        \hline
        sc\_type\_arc\_pos\_var\_perm  & 0x8D0 & 2256  \\
        \hline
\end{tabular}
    \caption{Базовые типы sc-элементов в json-формате}
    \label{basic_types}
\end{figure}

Остальные виды sc-элементов получаются комбинированием характеристик из этой таблицы с помощью операции логического ИЛИ. Например, переменная унарная дуга может быть получена комбинацией кода унарной дуги sc\_type\_uedge\_common и кода переменности sc\_type\_var:
\[sc\_type\_uedge\_common\_var = sc\_type\_uedge\_common | sc\_type\_var\]
Что будет равно сумме десятичных кодов элементов: \(4 + 64 = 68\).

\subsection{Полная таблица поддерживаемых типов}
На данный момент jmantic поддерживает 48 типов sc-элементов. 
\begin{figure}[H]
    \rowcolors{2}{gray!10}{white}
    % \rowcolors{1}{gray}{white}
    \centering
    \begin{tabular}{  | p{50mm} | c | }
    \rowcolor{gray!50}
		\hline
		Тип элемента & Десятичный код элемента \\
        \hline
		EdgeUCommon & 4 \\
		\hline
		EdgeDCommon & 8 \\
		\hline
		EdgeUCommonConst & 36 \\
		\hline
		EdgeDCommonConst & 40 \\
		\hline
		EdgeUCommonVar & 68 \\
		\hline
		EdgeDCommonVar & 72 \\
		\hline
		EdgeAccess & 16 \\
		\hline
		EdgeAccessConstPosPerm & 2224 \\
		\hline
		EdgeAccessConstNegPerm & 2352 \\
		\hline
		EdgeAccessConstFuzPerm & 2608 \\
		\hline
		EdgeAccessConstPosTemp & 1200 \\
		\hline
		EdgeAccessConstNegTemp & 1328 \\
		\hline
		EdgeAccessConstFuzTemp & 1584 \\
		\hline
		EdgeAccessVarPosPerm & 2256 \\
		\hline
		EdgeAccessVarNegPerm & 2384 \\
		\hline
		EdgeAccessVarFuzPerm & 2640 \\
		\hline
		EdgeAccessVarPosTemp & 1232 \\
		\hline
		EdgeAccessVarNegTemp & 1360 \\
		\hline
		EdgeAccessVarFuzTemp & 1616 \\
		\hline
		Const & 32 \\
		\hline
		Var & 64 \\
		\hline
		Node & 1 \\
		\hline
		Link & 2 \\
		\hline
		Unknown & undefined \\
		\hline
		NodeConst & 33 \\
		\hline
		NodeVar & 65 \\
		\hline
		LinkConst & 34 \\
		\hline
		LinkVar & 66 \\
		\hline
		NodeStruct & 257 \\
		\hline
		NodeTuple & 129 \\
		\hline
		NodeRole & 513 \\
		\hline
		NodeNoRole & 1025 \\
		\hline
		NodeClass & 2049 \\
		\hline
		NodeAbstract & 4097 \\
		\hline
		NodeMaterial & 8193 \\
		\hline
		NodeConstStruct & 289 \\
		\hline
		NodeConstTuple & 161 \\
		\hline
		NodeConstRole & 545 \\
		\hline
		NodeConstNoRole & 1057 \\
		\hline
		NodeConstClass & 2081 \\
		\hline
		NodeConstAbstract & 4129 \\
		\hline
		NodeConstMaterial & 8225 \\
		\hline
		NodeVarStruct & 321 \\
		\hline
		NodeVarTuple & 193 \\
		\hline
		NodeVarRole & 577 \\
		\hline
		NodeVarNoRole & 1089 \\
		\hline
		NodeVarClass & 2113 \\
		\hline
		NodeVarAbstract & 4161 \\
		\hline
		NodeVarMaterial & 8257 \\
		\hline
\end{tabular}
    \caption{Список всех типов sc-элементов}
    \label{all_types}

\end{figure}
